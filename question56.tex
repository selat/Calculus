\colquestion{Теорема о дифференцировании ИЗОП ФКП и замечание к ней.}
%% \begin{plan}
%% \item $z = x + iy, \omega = u + iv$. Выражаем $u, v$.
%% \end{plan}

\begin{col-answer-preambule}
\end{col-answer-preambule}
Из интегральной формулы Коши следует
\begin{align*}
  \intl_{\partial D}\dfrac{f(t)}{t - z}dt =
  \begin{cases}
    2\pi if(z), z \in D,\\
    0, z \notin D,
  \end{cases}
\end{align*}
В данном случае имеем ИЗОП от ФКП вида
\begin{equation}
  \label{eq:lecture32-03}
  F(z) = \intl_l\phi(t, z)dt,
\end{equation}
где $z \in D$ в плоскости \circled{$z$}, а $l$ - некоторая линия в плоскости \circled{$t$}.

В зависимости от свойств подинтегральной функции в \eqref{eq:lecture32-03} исследуем соответствующие
свойства ИЗОП \eqref{eq:lecture32-03}.

\begin{theorem}[О дифференцировании ИЗОП ФКП]
  Пусть для $\forall t \in l$ функция $\phi(t, z)$ - аналитическая по $z \in G \subset \mathbb{C}$ -
  область, причём у $\phi(t, z)$ её производная $\dfrac{\partial \phi(t, z)}{\partial}$непрерывна,
  как по $z \in G$, так и по $t \in l$.

  Тогда ИЗОП \eqref{eq:lecture32-03} является аналитической функцией в области $G$, производная от
  которой вычисляется по правилу Лейбница
  \begin{equation}
    \label{eq:lecture32-04}
    F'(z) = \intl_l\dfrac{\partial\phi(t, z)}{\partial z}dt.
  \end{equation}
\end{theorem}
\begin{proof}
  Рассмотрим
  \begin{align*}
    x  = \operatorname{Re}z, y = \operatorname{Im}z, \tau = \operatorname{Re}t,
    s = \operatorname{Im}t, u = \operatorname{Re}\phi(t, z), v = \operatorname{Im}\phi(t, z),
  \end{align*}
  по формуле вычисления интеграла ФКП через КРИ-2 имеем
  \begin{align*}
    &F(z) = \sqcase{z = x + iy, t = \tau + is, \phi=  u(\tau, s, x, y) + iv(\tau, s, x, y)} =
    \intl_l(u + iv)(d\tau + ids) = \ldots = H(x, y) + iR(x, y),\\
    &\text{Где } H(x, y) = \intl_lud\tau - vds, R(x, y) = \intl_lvd\tau + uds.
  \end{align*}
  Для обоснования аналитичности $F(z)$ нужно показать, что функции
  \begin{align*}
    \begin{cases}
      H = H(x, y),\\
      R = R(x, y),\\
    \end{cases}
  \end{align*}
  удовлетворяют условию Коши-Римана
  \begin{align*}
    \begin{cases}
      H'_x = R'_y,\\
      H'_y = -R'_x.
    \end{cases}
  \end{align*}
  Нетрудно видеть, что интегральное представление от $H(x, y)$ и $R(x, y)$ позволяет использовать
  теорему о почленном дифференцировании ИЗОП, записанную через КРИ-2.

  В силу этого имеем по правилу Лейбница дфференцирования ИЗОП
  \begin{enumerate}
  \item $\exists H'_x = \intl_lu'_xd\tau - v'_xds; \exists R'_y = \intl_lv'_yd\tau + u'_yds$.
    В силу аналитичности $\phi(t, s)$ имеем
    \begin{align*}
      \begin{cases}
        u'_x = v'_y,\\
        v'_x = -u'_y,
      \end{cases}\text{ т.е. }
      H'_x = \sqcase{u'_x = v'_x\\u'_y = -v'_x} = \intl_lv'_yd\tau + u'_yds = R'_y.
    \end{align*}
  \item $\begin{aligned}[t]
    \exists H'_y = \intl_lu'_yd\tau - v'_yds = \sqcase{u'_y = -v'_x\\v'_y = u'_x} =
    -\intl_lv'_xd\tau + u'_xds = -R'_x.
  \end{aligned}$
  \end{enumerate}
  Т.к. для $F(z) = H + iR$ выполняется усовие Коши-Римана, то эта ФКП аналитична, при этом для её
  производной получаем
  \begin{align*}
    F'(z) = H'_x + iR'_x = \intl_lu'_xd\tau - v'_xds + i\intl_lv'_xd\tau + u'_xds = \ldots =
    \intl_l\dfrac{\partial \phi(t, z)}{\partial z}dt.
  \end{align*}
\end{proof}
\begin{note}
  При выводе \eqref{eq:lecture32-04} предполагалось, что линия $l$ - ограничена. Доказательство
  сохраняется и когда $l$ неоограниченна, т.е. когда \eqref{eq:lecture32-03} - НИЗОП ФКП. В этом
  случае также справедлива формула \eqref{eq:lecture32-04}.
\end{note}
