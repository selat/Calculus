\begin{col-answer-preambule}
\end{col-answer-preambule}

\colquestion{Теорема о локальной равномерной сходимости СтР, замечания к ней и следствие из неё (о равенстве степенных рядов).}
\begin{plan}
\item Рассматриваем произвольный отрезок из интеравала сходимости.
\item Делаем отрезок симметричным относительно $x_0$.
\item Ограничиваем члены СтР сверху: $a_n r^n$.
\item Применяем ообобщённый признак Коши (супремум пределов)
\end{plan}
Следствие о равенстве СтР:

\begin{plan}
\item Приравниваем сумму двух рядов
\item Подставляем $x_0$, получаем равенство $a_0 = b_0$
\item Делим остаток на $(x - x_0)$
\item Предел $x \to x_0$, получаем равенство $a_1 = b_1$. Goto 2.
\end{plan}
\begin{theorem}[о локальной равномерной сходимости СтР]
	Если СтР \eqref{2.01} имеет ненулевой радиус сходимости, то этот ряд \eqref{2.01}  сходится равномерно на любом отрезке из интервала сходимости данного ряда.
\end{theorem}
\begin{proof}$  $

	Рассмотрим $ \forall \; [a, b] \subset I = \interval]{ \nullFrac x_0-R \;;\; x_0 + R \nullFrac}[$, где $ R > 0 $ - радиус сходимости СтР \eqref{2.01}. Имеем:
	\begin{equation}
	\label{2.10}
	x_0-R < a < b < x+R \Rightarrow -R < a-x_0 < b-x_0 < R \Rightarrow
	\begin{cases}
	\abs{a-x_0} < R, \\
	\abs{b-x_0} < R.
	\end{cases}
	\end{equation}
	Полагая $ r = \max \left\{\nullFrac \abs{a-x_0}, \; \abs{b-x_0} \nullFrac \right\} $, в силу \eqref{2.10} получаем:
	\begin{equation}
	\label{2.11}
	0 \leq r < R.
	\end{equation}
	Отсюда для $ \forall \; x \in [a, b] $ получаем:
	\begin{equation*}
	\abs{x - x_0} \leq \max \left\{\nullFrac \abs{a-x_0}, \; \abs{b-x_0} \nullFrac \right\} = r,
	\end{equation*}
	поэтому для $ \forall \; n \in \mathbb{N}_0 $ имеем:
	\begin{equation*}
	\abs{a_n (x-x_0)^n} = \abs{a_n} \abs{x-x_0}^n \leq \abs{a_n} r^n = c_n
	\text{ - мажоранта.}
	\end{equation*}

	Применяя к ряду $ c_n $ обобщённый признак Коши сходимости ЧР, получаем:
	\begin{equation*}
	\exists \; \overline{\limninf} \sqrt[n]{\abs{c_n}} =
	\overline{\limninf} \sqrt[n]{\abs{a_n} r^n} =
	r \cdot \underbrace{\overline{\limninf} \sqrt[n]{\abs{a_n}}}_{\frac{1}{R}}
	\overset{\eqref{2.09}}{=} \dfrac{r}{R} \overset{\eqref{2.11}}{<} 1,
	\end{equation*}
	а значит, ряд $ \sum c_n $ сходится.\\

	Таким образом, мы получили равномерно сходящуюся числовую мажоранту, и поэтому, по мажорантному признаку Вейерштрасса для ФР, рассматриваемый СтР \eqref{2.01} будет равномерно сходиться на
	$ \forall \; [a, b] \subset I$.
\end{proof}

\begin{notes}
	\item Из доказанной теоремы следует, что любой СтР сходится локально равномерно на интервале своей сходимости.

	\item Применяя теорему Стокса-Зейделя для ФР и учитывая, что в \eqref{2.01} все слагаемые являются непрерывными функциями на $ I $,
	в силу локальной равномерной сходимости \eqref{2.01} на $ I $, внутри интервала сходимости сумма любого СтР \eqref{2.01}  будет являться непрерывной функцией.
\end{notes}
\begin{consequence}[о равенстве СтР]
	Если для СтР \eqref{2.01} с непрерывной суммой $ S_n(x) $ есть степенной ряд
	$ \sumnzi b_n (x-x_0)^n $ с соответствующей суммой $ T(x) $, причём $ T(x) = S(x) $ в некоторой окрестности центра разложения \nolinebreak$ x_0 $, то тогда и сами СтР совпадают,
	т.е. $ a_n = b_n, \; \text{для } \; \forall \; n \in \mathbb{N}_0 $.
\end{consequence}

\begin{proof} Пусть имеем, что
	\begin{equation*}
	S(x) = a_0 + a_1 (x-x_0) + \ldots = b_0 + b_1 (x-x_0) + \ldots = T(x).
	\end{equation*}

	В силу непрерывности $ S(x) $ и $ T(x) $ в соответствующей окрестности точки $ x_0 $ при
	$ x \to x_0 $, получаем:
	\begin{equation*}
	\begin{split}
	& a_0 = \lim\limits_{x\to x_0} S(x) = \lim\limits_{x\to x_0} T(x) = b_0, \text{ отсюда }\\
	& a_1 (x-x_0) + a_2(x-x_0)^2 + \ldots = b_1 (x-x_0) + b_2(x-x_0)^2 + \ldots.
	\end{split}
	\end{equation*}
	Таким образом, для $ \forall \; x \neq x_0 $ имеем:
	\begin{equation*}
	a_1 + a_2(x-x_0)  + \ldots =  b_1 + b_2(x-x_0) + \ldots.
	\end{equation*}

	Используя опять соответствующую окрестность точки $ x_0 $, при $ x \to x_0 $, получим, что
	$ a_1 = b_1 $ и так далее (по ММИ).
\end{proof}
