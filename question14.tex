\begin{col-answer-preambule}
		Аналогично, как и теорема о замене переменных в НИ-2, обосновываются формулы двойной подстановки (аналог формулы Ньютона-Лейбница) и метод интегрирования по частям для НИ-2 и НИ-1.
\end{col-answer-preambule}

\colquestion{Формула двойной подстановки для НИ и интегрирование по частям в НИ.}
\important{Формула Ньютона-Лейбница для НИ.}

Пусть для простоты для $f(x)$, определённой для $\forall \; x \in [a, b[$, где $b = + \infty$ или $f(b - 0) = \infty$ существует дифференцируемая первообразная $F(x)$, т.е. $\exists \; F^{'}(x) = f(x), \text{для }\forall \; x \in [a, b[$. Тогда имеем:
\begin{equation*}
\begin{split}
&\dint\limits_{a}^{b} f(x)dx = \lim\limitslimits{c \to + \infty, b \to + \infty}{(c \to b-0,b \to -0)} \dint\limits_{a}^{c} f(x)dx = \lim\limits_{c\to b-0} \begin{sqcases} F(x) \end{sqcases}^{c}_{a} = \\
& =\lim\limits_{c\to b-0} \left(F(c) - F(a)\right) = F(b-0) - F(a) = \begin{sqcases} F(x) \end{sqcases}^{b-0}_{a}.
\end{split}
\end{equation*}
При этом исходный интеграл сходится тогда и только тогда, когда значения $\linebreak F(b-0), F(+\infty)$ конечны.

На практике формулы двойной подстановки используются в том же виде, что и для ОИ: $\dint\limits_{a}^{b} f(x)dx = \begin{sqcases} \dint\limits f(x)dx \end{sqcases}^b_a$.
\important{Интегрирование по частям в НИ.}

Пусть $u = u(x), v = v(x)$ определены для $\forall \; x \in [a; b[$ и $f(b) = \infty$.

Если существует конечный предел $\lim\limitslimits{x \to b-0}{(x\to +\infty)} u(x) v(x) = u(b-0)v(b-0) \in \mathbb{R}$.

В случае сходимости одного из использованных ниже интегралов, получаем:
\begin{equation*}
\dint\limits_{a}^{b} u(x) v^{'}(x)dx = \begin{sqcases} u(x)v(x)\end{sqcases}^b_a - \dint\limits_{a}^{b} u^{'}(x) v(x)dx.
\end{equation*}

На практике удобнее использовать:
\begin{equation*}
\dint\limits_{a}^{b} udv = \begin{sqcases} uv \end{sqcases}^b_a - \dint\limits_{a}^{b} vdu.
\end{equation*}