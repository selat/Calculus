\colquestion{Дифференцируемые ФКП. Критерий Коши-Римана дифференцируемости ФКП и замечания к нему.}
%% \begin{plan}
%% \item $z = x + iy, \omega = u + iv$. Выражаем $u, v$.
%% \end{plan}

\begin{col-answer-preambule}
\end{col-answer-preambule}

Для получения условия дифференцируемости ФКП через её действительную $u = \operatorname{Re}f(z)$ и
мнимую $v = \operatorname{Im}f(z)$ части будем использовать \important{условие Коши-Римана}
\begin{align*}
  \begin{cases}
    u = u(x, y),
    v = v(x, y),
  \end{cases}
\end{align*}

считаются удовлетворяющими условию Коши-Римана, если у них существуют соответствующие производные
первого порядка, для которых выполняется
\begin{equation}
  \begin{cases}
    \label{eq:lecture16-08}
    u'_x = v'_y\\
    u'_y = -v'_x\\
  \end{cases}
\end{equation}


\begin{theorem}[Критерий Коши-Римана дифференцируемости ФКП]
  ФКП $f(z) = u + iv$ дифференцируема в области $G \subset D(f) \Leftrightarrow
  u = \operatorname{Re}f(z), v = \operatorname{Im}f(z)$ удовлетворяют условию Коши-Римана
  \eqref{eq:lecture16-08}.
\end{theorem}
\begin{proof}
  \circled{$\Rightarrow$} Пусть $f(z)$ - дифференцируема в $G$. Тогда $\forall z \in G \Rightarrow
  \exists f'(z) \in \mathbb{C}$, т.е.
  \begin{align*}
    \exists\liml_{\Delta z \to 0}\dfrac{\Delta f(z)}{\Delta z} = \sqcase{
      f = u + iv,
      \Delta f = \Delta u + i\Delta v,
      \Delta u = u(x + \Delta x, y + \Delta y) - u(x, y),
      \Delta v = v(x + \Delta x, y + \Delta y) - v(x, y)} = f'(z) \in \mathbb{C}.
  \end{align*}
  В силу любого допустимого $\Delta z = \Delta x + i \Delta y$ при $\Delta z \to 0 \Rightarrow
  \Delta x \to 0, \Delta y \to 0$, а тогда из того, что для Ф2П из существования двойного предела
  следует существование соответствующих частных пределов, при использовании приращений вдоль
  координатных осей получаем
  \begin{enumerate}
  \item $\begin{aligned}[t]
    \Delta y = 0, \Delta x \to 0 \Rightarrow f'(z) = \liml_{\Delta x \to 0}
    \dfrac{u(x + \Delta x, y) - u(x, y) + i(v(x + \Delta x, y) - v(x, y))}{\Delta x} =
    u'_x + iv'_x.
  \end{aligned}$
  \item $\begin{aligned}[t]
    \Delta x = 0, \Delta y \to 0 \Rightarrow f'(z) = \liml_{\Delta y \to 0}
    \dfrac{u(x, y + \Delta y) - u(x, y) + i(v(x, y + \Delta y) - v(x, y))}{i\Delta y} =
    v'_y - iu'_y.
  \end{aligned}$
  \end{enumerate}

  $f(z) = u'_x + iv'_x = v'_y - iu'_y$. Из равенства получаем
  \begin{equation}
    \begin{cases}
      u'_x = v'_y\\
      u'_y = -v'_x\\
    \end{cases} \Leftrightarrow \eqref{eq:lecture16-08}
  \end{equation}

  \circled{$\Leftarrow$} Для простоты будем считать, что для Ф2П
  \begin{align*}
    \begin{cases}
      u = \operatorname{Re}f(z),\\
      v = \operatorname{Im}f(z),
    \end{cases}
  \end{align*}
  существуют не только частные производные, удовлетворяющие \eqref{eq:lecture16-08}, но и что эти
  производные непрерывны, т.е. используемые функции непрерывно дифференцируемы, а тогда
  соответствующие их приращения записываются в виде
  \begin{align*}
    &\Delta u(x, y) = u(x + \Delta x, y + \Delta y) - u(x, y) = u'_x\Delta x + u'_y \Delta y +
    o(\sqrt{\Delta x^2 + \Delta y^2})\\
    &\Delta v(x, y) = v(x + \Delta x, y + \Delta y) - v(x, y) = v'_x\Delta x + v'_y \Delta y +
    o(\sqrt{\Delta x^2 + \Delta y^2})\\
  \end{align*}
  Отсюда, для
  \begin{align*}
    &\Delta f(z) = f(z + \Delta z) - f(z) = \Delta u + i \Delta v \Rightarrow\\
    &\Rightarrow \Delta f(z) = u'_x\Delta x + u'_y \Delta y + o(\sqrt{\Delta x^2 + \Delta y^2}) +
    i\parenthesis{v'_x\Delta x + v'_y \Delta y + o(\sqrt{\Delta x^2 + \Delta y^2})} =\\
    &=(u'_x + iv'_x)\Delta x + (u'_y + iv'_y)\Delta y + o(\sqrt{\Delta x^2 + \Delta y^2}) =
    \sqcase{
      u'_x = v'_y\\
      u'_y = -v'_x} =\\
    &=(v'_y + iv'_x)\Delta x + (-v'_x + iv'_y)\Delta y + o(\sqrt{\Delta x^2 + \Delta y^2}) = \\
    &=(\Delta x + i\Delta y)v'_y + (i\Delta x - \Delta y)v'_x + o(\sqrt{\Delta x^2 + \Delta y^2}) =\\
    &=(\Delta x + i\Delta y)v'_y + i(i\Delta x + i\Delta y)v'_x + o(\sqrt{\Delta x^2 + \Delta y^2})=\\
    &= (\Delta x + i\Delta y)(v'_y + iv'_x) + o(\sqrt{\Delta x^2 + \Delta y^2}) =
    \sqcase{\Delta x + i\Delta y = \Delta z,\\
      \abs{\Delta z} = \sqrt{\Delta x^2 + \Delta y^2}} =
    (v'_y + iv'_x)\Delta z + \gamma, \text{ где } \gamma = o(\abs{\Delta z}), \Delta \to 0.
  \end{align*}
  Имеем
  \begin{align*}
    \abs{\dfrac{\gamma}{\Delta z}} \overset{\Delta z \neq 0}{=}
    \abs{\dfrac{o(\abs{\Delta z})}{\abs{\Delta z}}} \xrightarrow[\Delta z \to 0]{} 0, \text{ т.е. }
    \gamma = o(\Delta z), \Delta z \to 0.
  \end{align*}
  Из полученного представления $\Delta f(z) = (v'_y + iv'_x)\Delta z + o(\Delta z), \Delta z \to 0$,
  получаем в силу определения дифференцируемость $f(z)$. Её производную, в частности, можно
  вычислить по формуле $f'(z) = v'_y + iv'_x$.
\end{proof}
\begin{notes}
\item При обосновании достаточности существенную роль играло предположение о непрерывной
  дифференцируемости $u = \operatorname{Re}f(z)$ и $v = \operatorname{Im}f(z)$. В общем случае
  можно показать, что это предположение излишне, но тогда доказательство значительно усложняется.
\item Если у нас $f(z)$ дифференцируема в $G \subset D(f)$, то $f(z)$ будет бесконечное число
  раз дифференцируема в $G$, т.е. $\exists f'(z)$ - .. для ФКП и существование всех остальных
  производных высших порядков.
\item Из доказательства теоремы и условия Коши-Римана \eqref{eq:lecture16-08} следует, что
  производную ФКП можно вычислить через её действительную и мнимую части, используя одну из
  формул
  \begin{align*}
    f'(z) = u'_x + iv'_x = u'_x - iu'_y = v'_y + iv'_x = v'_y - iu'_y.
  \end{align*}
\end{notes}
