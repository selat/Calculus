\colquestion{Теорема о тригонометрическом многочлене наименьшего отклонения.}

\begin{col-answer-preambule}
\end{col-answer-preambule}

\begin{equation}
  \label{eq:lecture06-07}
  T_n(x) = \dfrac{A_0}{2} + \suml_{k = 1}^n(A_k\cos kx + B_k\sin kx).
\end{equation}

\begin{equation}
  \label{eq:lecture06-08}
  A_m = \dfrac{1}{\pi}\intl_{-\pi}^{\pi}T_n(x)\cos mx dx, m = \overline{0, n};
\end{equation}

\begin{equation}
  \label{eq:lecture06-09}
  B_m = \dfrac{1}{\pi}\intl_{-\pi}^{\pi}T_n(x)\sin mx dx, m = \overline{1, n};
\end{equation}

\begin{equation}
  \label{eq:lecture06-10}
  \dfrac{1}{\pi}\intl_{-\pi}^{\pi}T_n^2(x)dx = \dfrac{A_0^2}{2} + \suml_{k = 1}^n(A_k^2 + B_k^2).
\end{equation}

\begin{equation}
  \Delta = \norm{f(x) - T_n(x)} = \parenthesis{\intl_{-\pi}^{\pi}(f(x) - T_n(x))^2dx}^{\frac{1}{2}}
  \label{eq:lecture06-11}
\end{equation}

\begin{theorem}[О ТМ наименьшего отклонения]
  Среди ТМ \eqref{eq:lecture06-07} фиксированной степени $\leq n$ многочлен Фурье для $f(x)$ на
  $\interval[{-\pi; \pi}]$ - многочлен
  \begin{equation}
    S_n(x) = \dfrac{a_0}{2} + \suml_{k = 1}^n(a_k\cos kx + b_k\sin kx),
  \end{equation}
  коэффициенты которого вычисляются по формулам
  \begin{align}
    a_m = \dfrac{1}{\pi}\intl_{-\pi}^{\pi}f(x)\cos mx dx, m = \overline{0, n};\\
    b_m = \dfrac{1}{\pi}\intl_{-\pi}^{\pi}f(x)\sin mx dx, m = \overline{1, n};\\
  \end{align}
\end{theorem}
\begin{proof}
  Рассматривая произвольный тригонометрический многочлен \eqref{eq:lecture06-07} $\fix$ степени
  $\leq n$ и используя формулы \eqref{eq:lecture06-08}, \eqref{eq:lecture06-09},
  \eqref{eq:lecture06-10}, для отклонения \eqref{eq:lecture06-11} имеем
  \begin{align*}
    &\Delta = \parenthesis{\norm{f(x) - T_n(x)}} = \intl_{-\pi}^{\pi}(f(x) - T_n(x))^2dx =
    \intl_{-\pi}^{\pi}f^2(x)dx - 2\intl_{-\pi}^{\pi}f(x)T_n(x)dx + \intl_{-\pi}^{\pi}T_n^2(x)dx =\\
    &=\intl_{-\pi}^{\pi}f^2(x)dx - 2\parenthesis{\dfrac{A_0}{2}\intl_{-\pi}^{\pi}f(x)dx +
      \suml_{k = 1}^n\parenthesis{A_k\intl_{-\pi}^{\pi}f(x)\cos kxdx + B_k \intl_{-\pi}^{\pi}f(x)\sin kxdx}}
    + \pi\parenthesis{\dfrac{A_0^2}{2} + \suml_{k = 1}^n(A_k^2 + B_k^2)} =\\
    &=\intl_{-\pi}^{\pi}f^2(x)dx - 2\pi\parenthesis{\dfrac{a_0A_0}{2}  + \suml_{k = 1}^n(a_kA_k + b_kB_k)}
    + \pi\parenthesis{\dfrac{A_0^2}{2} + \suml_{k = 1}^n(A_k^2 + B_k^2)}=\\
    &=\intl_{-\pi}^{\pi}f^2(x)dx + \dfrac{\pi}{2}\parenthesis{A_0^2 - a_0A_0
      + 2\suml_{k = 1}^n(A_k^2 - 2a_kA_k)} + \pi\suml_{k = 1}^n(B_k^2 - 2b_kB_k)=\\
      &=\intl_{-\pi}^{\pi}f^2(x)dx + \dfrac{\pi}{2}(A_0 - a_0)^2
      + \pi\suml_{k = 1}^n(A_k - a_k)^2 + \pi\suml_{k = 1}^n(B_k - b_k)^2
          - \dfrac{\pi}{2}\parenthesis{a_0^2 + 2\suml_{k = 1}^na_k^2 + 2\suml_{k = 1}^nb_k^2} \geq\\
          &\geq \sqcase{\forall A_k \in \R{}, k = \overline{0, n},\\
            \forall B_k \in \R{}, k = \overline{1, n},} \geq
          \intl_{-\pi}^{\pi}f^2(x)dx - \pi\parenthesis{a_0^2 + 2\suml_{k = 1}^na_k^2
            + 2\suml_{k = 1}^nb_k^2}.
  \end{align*}
  RHS не зависит от выбора $T_n(x)$ и минимум достигается при
  \begin{align*}
    &\forall A_k = a_k, k = \overline{0, n},\\
    &\forall B_k = b_k, k = \overline{1, n},
  \end{align*}
\end{proof}
